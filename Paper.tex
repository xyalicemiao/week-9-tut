% Options for packages loaded elsewhere
\PassOptionsToPackage{unicode}{hyperref}
\PassOptionsToPackage{hyphens}{url}
\PassOptionsToPackage{dvipsnames,svgnames,x11names}{xcolor}
%
\documentclass[
  letterpaper,
  DIV=11,
  numbers=noendperiod]{scrartcl}

\usepackage{amsmath,amssymb}
\usepackage{iftex}
\ifPDFTeX
  \usepackage[T1]{fontenc}
  \usepackage[utf8]{inputenc}
  \usepackage{textcomp} % provide euro and other symbols
\else % if luatex or xetex
  \usepackage{unicode-math}
  \defaultfontfeatures{Scale=MatchLowercase}
  \defaultfontfeatures[\rmfamily]{Ligatures=TeX,Scale=1}
\fi
\usepackage{lmodern}
\ifPDFTeX\else  
    % xetex/luatex font selection
\fi
% Use upquote if available, for straight quotes in verbatim environments
\IfFileExists{upquote.sty}{\usepackage{upquote}}{}
\IfFileExists{microtype.sty}{% use microtype if available
  \usepackage[]{microtype}
  \UseMicrotypeSet[protrusion]{basicmath} % disable protrusion for tt fonts
}{}
\makeatletter
\@ifundefined{KOMAClassName}{% if non-KOMA class
  \IfFileExists{parskip.sty}{%
    \usepackage{parskip}
  }{% else
    \setlength{\parindent}{0pt}
    \setlength{\parskip}{6pt plus 2pt minus 1pt}}
}{% if KOMA class
  \KOMAoptions{parskip=half}}
\makeatother
\usepackage{xcolor}
\setlength{\emergencystretch}{3em} % prevent overfull lines
\setcounter{secnumdepth}{-\maxdimen} % remove section numbering
% Make \paragraph and \subparagraph free-standing
\ifx\paragraph\undefined\else
  \let\oldparagraph\paragraph
  \renewcommand{\paragraph}[1]{\oldparagraph{#1}\mbox{}}
\fi
\ifx\subparagraph\undefined\else
  \let\oldsubparagraph\subparagraph
  \renewcommand{\subparagraph}[1]{\oldsubparagraph{#1}\mbox{}}
\fi


\providecommand{\tightlist}{%
  \setlength{\itemsep}{0pt}\setlength{\parskip}{0pt}}\usepackage{longtable,booktabs,array}
\usepackage{calc} % for calculating minipage widths
% Correct order of tables after \paragraph or \subparagraph
\usepackage{etoolbox}
\makeatletter
\patchcmd\longtable{\par}{\if@noskipsec\mbox{}\fi\par}{}{}
\makeatother
% Allow footnotes in longtable head/foot
\IfFileExists{footnotehyper.sty}{\usepackage{footnotehyper}}{\usepackage{footnote}}
\makesavenoteenv{longtable}
\usepackage{graphicx}
\makeatletter
\def\maxwidth{\ifdim\Gin@nat@width>\linewidth\linewidth\else\Gin@nat@width\fi}
\def\maxheight{\ifdim\Gin@nat@height>\textheight\textheight\else\Gin@nat@height\fi}
\makeatother
% Scale images if necessary, so that they will not overflow the page
% margins by default, and it is still possible to overwrite the defaults
% using explicit options in \includegraphics[width, height, ...]{}
\setkeys{Gin}{width=\maxwidth,height=\maxheight,keepaspectratio}
% Set default figure placement to htbp
\makeatletter
\def\fps@figure{htbp}
\makeatother

\KOMAoption{captions}{tableheading}
\makeatletter
\@ifpackageloaded{caption}{}{\usepackage{caption}}
\AtBeginDocument{%
\ifdefined\contentsname
  \renewcommand*\contentsname{Table of contents}
\else
  \newcommand\contentsname{Table of contents}
\fi
\ifdefined\listfigurename
  \renewcommand*\listfigurename{List of Figures}
\else
  \newcommand\listfigurename{List of Figures}
\fi
\ifdefined\listtablename
  \renewcommand*\listtablename{List of Tables}
\else
  \newcommand\listtablename{List of Tables}
\fi
\ifdefined\figurename
  \renewcommand*\figurename{Figure}
\else
  \newcommand\figurename{Figure}
\fi
\ifdefined\tablename
  \renewcommand*\tablename{Table}
\else
  \newcommand\tablename{Table}
\fi
}
\@ifpackageloaded{float}{}{\usepackage{float}}
\floatstyle{ruled}
\@ifundefined{c@chapter}{\newfloat{codelisting}{h}{lop}}{\newfloat{codelisting}{h}{lop}[chapter]}
\floatname{codelisting}{Listing}
\newcommand*\listoflistings{\listof{codelisting}{List of Listings}}
\makeatother
\makeatletter
\makeatother
\makeatletter
\@ifpackageloaded{caption}{}{\usepackage{caption}}
\@ifpackageloaded{subcaption}{}{\usepackage{subcaption}}
\makeatother
\ifLuaTeX
  \usepackage{selnolig}  % disable illegal ligatures
\fi
\usepackage{bookmark}

\IfFileExists{xurl.sty}{\usepackage{xurl}}{} % add URL line breaks if available
\urlstyle{same} % disable monospaced font for URLs
\hypersetup{
  pdftitle={Letting Data Speaking for Themselves},
  pdfauthor={Xiaoyu (Alice) Miao},
  colorlinks=true,
  linkcolor={blue},
  filecolor={Maroon},
  citecolor={Blue},
  urlcolor={Blue},
  pdfcreator={LaTeX via pandoc}}

\title{Letting Data Speaking for Themselves}
\author{Xiaoyu (Alice) Miao}
\date{}

\begin{document}
\maketitle

The notion of letting ``data speak for themselves'' is appealing in its
simplicity and purported objectivity. However, this perspective, when
scrutinized through the lenses provided by (\textbf{Jordan\_2020?}),
(\textbf{D?})`Ignazio\_and\_Klein\_2020, and (\textbf{Au\_2020?}),
reveals a landscape where data cannot and should not be expected to
stand alone as arbiters of truth. Their arguments collectively
underscore the complexities involved in data generation, processing, and
analysis, challenging the feasibility and desirability of adopting a
hands-off approach to data interpretation. This essay explores the
extent to which we should allow data to speak for themselves, drawing on
these authors' insights and considering broader implications for data
science and ethics.

\textbf{The Myth of Unmediated Data}

At the heart of the discussion is the misconception that data, in their
raw form, are neutral and unbiased reflections of reality.
(\textbf{Jordan\_2019?}) dismantles this notion by highlighting the
entanglement of data with human decisions, biases, and societal
structures. The very act of data collection involves choices about what
to measure, how to measure it, and which data to record or discard.
These decisions, often invisible in the final dataset, imbue the data
with the perspectives and prejudices of those who collect and process
it.

(\textbf{D?})'Ignazio\_and\_Klein\_2020 extend this argument through the
lens of intersectional feminism, emphasizing how data practices can
reinforce or challenge existing power dynamics. Their critique of data
science as predominantly white and male-dominated not only questions
whose interests are served by data-driven decisions but also which
voices and experiences are marginalized or excluded. This perspective
calls for a more critical engagement with data, one that acknowledges
and seeks to rectify these imbalances.

(\textbf{Au\_2020?}), meanwhile, focuses on the process of data cleaning
as an analytical act that imposes interpretation on the dataset. He
argues that no data are truly raw; they are always already shaped by
human actions and decisions. The choices made during data
cleaning---what to exclude, normalize, or correct---are not merely
technical adjustments but analytical judgments that significantly affect
the outcome of data analysis. This view challenges the notion of data
neutrality and underscores the responsibility of data scientists to
engage deeply with their datasets, understanding their limitations and
biases.

\textbf{The Role of the Data Scientist}

The works of Jordan, D'Ignazio and Klein, and Au collectively argue for
a more active role of the data scientist in interpreting and presenting
data. Far from being passive conduits for data to speak, data scientists
are interpreters and storytellers who mediate between data and their
implications for the world. This role requires a deep understanding of
the data's origins, a critical awareness of one's own biases, and a
commitment to ethical principles that prioritize fairness, transparency,
and inclusivity.

Engaging with data in this way also involves recognizing the limits of
what data can tell us. Data are not self-explanatory; they require
context, theory, and human insight to yield meaningful conclusions. The
insistence on letting data speak for themselves risks oversimplifying
complex phenomena and overlooking the interpretive work necessary to
understand the underlying patterns and causes.

\textbf{Collaborative and Multidisciplinary Approaches to Data Science}

The articles collectively hint at the necessity of collaborative and
multidisciplinary approaches to data science. Jordan's discussion about
the need for a new engineering discipline that integrates social
sciences and humanities with computational methods suggests that
tackling complex data science challenges requires expertise beyond
traditional STEM fields. Similarly, D'Ignazio and Klein's application of
feminist theory to data science underscores the value of diverse
perspectives in uncovering and addressing biases in data practices. By
promoting interdisciplinary collaborations, data science can benefit
from a wider range of insights and methodologies, leading to more
innovative, inclusive, and ethically sound approaches to understanding
and acting on data. This aspect invites discussion on how different
disciplines can contribute to the development of data science and how
collaborative efforts can be fostered within academic, industry, and
policy-making contexts.

\textbf{Towards a More Responsible Data Practice}

The critique of the notion that data can speak for themselves leads to a
broader call for responsible data practice. This involves not only
technical proficiency but also ethical engagement with the societal
implications of data analysis. Data scientists must consider the
potential impact of their work on different communities, especially
those historically marginalized or vulnerable to harm. This includes
designing studies and algorithms that respect privacy, avoid
perpetuating biases, and strive for equitable outcomes.

Moreover, responsible data practice requires transparency about the
limitations and uncertainties of data analysis. Acknowledging these
limitations can foster a more nuanced understanding of what data can and
cannot tell us, encouraging a more humble and questioning approach to
data science.

\textbf{The Importance of Data Provenance and Transparency}

Drawing from Jordan's critique of the oversimplification of AI and data
science, as well as Au's emphasis on the intricacies of data cleaning,
another critical aspect to discuss is the importance of data provenance
and transparency. This involves understanding and documenting where data
come from, how they are collected, and any transformations they undergo
before analysis. Discussing data provenance and transparency can lead to
better awareness of the biases and limitations inherent in datasets,
thereby fostering more responsible use of data in scientific and
decision-making processes. It also underscores the need for clear
communication about the origins and manipulations of data, which is
essential for reproducibility and trust in data-driven findings.

\textbf{Ethical Considerations in Data Science Practices}

D'Ignazio and Klein's focus on intersectional feminism as a lens for
examining data science invites a broader discussion on the ethical
considerations that must guide data practices. This includes reflecting
on who benefits from data collection and analysis, who might be harmed,
and how the harms can be minimized. Ethics in data science also
encompasses issues of consent, privacy, and data security, especially in
contexts where the collection and use of data could lead to
surveillance, discrimination, or other forms of harm to individuals or
communities. Discussing ethical considerations demands that data
scientists not only follow legal requirements but also engage with
deeper moral questions about justice, equity, and respect for persons.

\textbf{Conclusion}

In conclusion, the works of (\textbf{Jordan\_2019?}),
(\textbf{DIgnazio\_and\_Klein\_2020?}), and (\textbf{Au\_2020?}) provide
compelling arguments against the notion of letting data speak for
themselves. They highlight the deeply human aspects of data science,
from the collection and preparation of data to their analysis and
interpretation. These insights call for a more engaged, critical, and
ethical approach to data, one that recognizes the power of data to shape
our understanding of the world and our impact on it. Far from being a
limitation, this approach can enrich data science, making it more
robust, equitable, and responsive to the complexities of human life.

\newpage

\#References



\end{document}
